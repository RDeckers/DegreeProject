\documentclass[a4paper]{report}
\begin{document}

\chapter{Strengths and Flaws of the Existing implementation}
The current implementation of AOD files is done using {\em ROOT}(CITE) objects. ROOT is a general-purpose (High-Energy) Physics framework extensively used throughout
all of CERNs experiments. Among its features is the ability to define classes as ROOT objects, granting them
features such as runtime-introspection, and store collections of ROOT objects in a general-purpose custom data format called a ROOT tree (TTree).
 The existing AOD format is defined as such a TTree and input and output operations are handled by ROOT.
 
\section{Strengths}

\section{Weaknesses}
Unfortunately, this approach does come with some downsides. Storing data in a TTree requires that the data is a ROOT object. In order to define a type as a ROOT object in c++
one has to inherit from the class {\em TObject}, and this causes problems. If a class derives from {\em TObject} it will inherit virtual functions, and in doing so it will no be a
Plain-Old-Data-type (POD-type), that is, it will have a virtual pointer meaning it can't be shared or copied between different nodes without doing some form of serialization to get rid of and
restore said pointer. In general, ROOT Trees where never designed to be shared across multiple nodes in a distributed fashion and once loaded from disk does not shy away from using
pointers internally.

Another downside is that one is dependent on the features and optimizations found in the generic ROOT Tree.  for example ROOT trees are optimized to handle files which are larger
than the available memory of a system but there is little control over this feature preventing further optimizations. dependency By building on top of ROOT Trees it also introduces a
where we are

finally, using the ROOT approach a collection of structs is defined and stored in a array-of-structs like fashion. As we will show later, this makes it hard to do efficient pruning
as different struct members will be interleaved together.

(not flat for one, bottlenecked, full of references, dead data from deep inheritance)
\end{document}
